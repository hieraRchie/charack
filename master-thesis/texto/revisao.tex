\chapter{Revis�o de Literatura} 
\label{capitulo:revisao}

\section{Mundos pseudo-infinitos}

-- falar aqui das zilhoes de abordagens para gera��o de mundos infinitos que n�s encontramos. Parece que todo mundo resolveu fazer pesquisa sobre isso de uma hora para outra.

-- Aqui vai a explica��o sobre o artigo frances sobre gera��o de cidades pseudo-infinitas (nosso artigo semente), aquela engine para gera��o de mundos virtuais, aquela outra ferramenta para gera��o de terrenos infinitos.

-- http://landscapestudio.omgames.co.uk/screenshots.html
-- http://www.vterrain.org/
-- http://www.pandromeda.com/products/mojoworldpro.php
-- http://wwwcg.in.tum.de/Research/Publications/FractalTerrain
-- http://www.howardzzh.com/research/terrain/
-- http://www.cs.brown.edu/~scd/world/home.html (exatamente o que queremos fazer)
-- http://www.earth3d.org/


\section{Fun��es de ru�do}

-- Como utilizamos muito fun��es de ruido e afins, falar aqui do Perlin e do Musgrave. Falar bastante do livro deles, porque � uma coisa que � bem relacionada com o nosso trabalho e que a gente vai utilizar bastante.

-- S� utilizamos o noise do Perlin aqui e o conceito de multi-fractal, mas sobre fractais eu vou falar depois.


\section{Fractais para gera��o de relevo}

-- Falar aqui das 3 formas que encontramos para gera��o de relevo atrav�s de fractais (deposi��o de sedimentos, altera��o do ponto m�dio e divis�o de n�o sei o que).

-- Falar tamb�m sobre os multifractais que tem no livro do Musgrave, que ele usa para gera��o de costas de continentes muito bonitas.

